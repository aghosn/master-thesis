% Chapter 3

\chapter{Theoretical Model} % TODO find another title

\label{Chapter3} % For referencing the chapter elsewhere, use \ref{Chapter3}

%----------------------------------------------------------------------------------------

% Define some commands to keep the formatting separated from the content 
\newcommand{\keyword}[1]{\textbf{#1}}
\newcommand{\tabhead}[1]{\textbf{#1}}
\newcommand{\code}[1]{\texttt{#1}}
\newcommand{\file}[1]{\texttt{\bfseries#1}}
\newcommand{\option}[1]{\texttt{\itshape#1}}

%---------------------------------------------------------------------------------------- 

Introduce Section by saying we focus on deoptimization for our model and GOAL.
\section{The pontential deoptimization targets}
\subsection{The Interpreter}
%requirements
%advantages
%disadvantages
\subsection{The Base function}
%requirements
%advantages
%disadvantages
\subsection{The Less Optimized function}
%requirements
%advantages
%disadvantages

\section{Versioning}
\subsection{A base function \& an instrumented base function}
%Base is LLVM IR -> not executed.
%Instrumented is such that entry points for each OSR exit, and all the OSR entries
%The order does not matter according to Rome dudes ... is that true ???
    %if yes then always possible to find an equivalent in another version

\subsection{Transitions between different versions}
%talk about the different trees i.e., chaining optimizations

\subsection{The space vs. time trade-off}
%keeping stuff around -> bad space
%generating everything when we need it -> bad time

\section{Our Model for Deoptimization}