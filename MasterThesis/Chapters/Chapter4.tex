% Chapter 3

\chapter{OSR in RJIT} % Main chapter title

\label{Chapter4} % For referencing the chapter elsewhere, use \ref{Chapter2} 

%----------------------------------------------------------------------------------------

% Define some commands to keep the formatting separated from the content 
\newcommand{\keyword}[1]{\textbf{#1}}
\newcommand{\tabhead}[1]{\textbf{#1}}
\newcommand{\code}[1]{\texttt{#1}}
\newcommand{\file}[1]{\texttt{\bfseries#1}}
\newcommand{\option}[1]{\texttt{\itshape#1}}

%----------------------------------------------------------------------------------------
\section{Overview}
\section{OSR RJIT mechanisms}
\subsection{OSR points and Conditions}
%simplified points
\subsection{OSRHandler}
%The patch points etc., the number of copies done by osr kit.
\subsubsection{Keeping versions}
%do the critic here for the too many versions
\subsubsection{Fixing the continuation function}
%deoptimization improving the performances.

\section{Prototypes}
\subsection{Transitive StateMaps}
\subsection{On the fly compilation}
