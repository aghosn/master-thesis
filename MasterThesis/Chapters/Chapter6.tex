% Chapter 6

\chapter{Conclusion} % Main chapter title

\label{Chapter6} % For referencing the chapter elsewhere, use \ref{Chapter6} 

%----------------------------------------------------------------------------------------

% Define some commands to keep the formatting separated from the content 
\newcommand{\keyword}[1]{\textbf{#1}}
\newcommand{\tabhead}[1]{\textbf{#1}}
\newcommand{\code}[1]{\texttt{#1}}
\newcommand{\file}[1]{\texttt{\bfseries#1}}
\newcommand{\option}[1]{\texttt{\itshape#1}}

%----------------------------------------------------------------------------------------

%We presented RJIT OSR, a prototype for on-stack-replacement deoptimization in RJIT.
%Showed how the basic OSR Kit library could be extended to better fit RJIT needs.
%Proved that re-using LLVM IR increases performance. 
%Overview of possible OSR techniques and prototyped extensions 
%Uncovered and prototyped possible extensions that RJIT might benefit from.
%Implemented a proof-of-concept inliner, breaking R semantics while preserving overall correctness of the program.
  

%Future work: better integration inside RJIT itself
    %Fresh IRs as part of the SEXP
    %Feedback on the code and recompilation with optimizations
    %Breaking R performance bottlenecks. 
We presented RJIT OSR, a prototype for runtime deoptimization in RJIT, based on the OSR Kit library\cite{OSRKit}.
We adapted the library to better fit the deoptimization case, and answered challenges particular to the RJIT project.
Re-using LLVM IR accros different different compilation modules proved to be three times more efficient than re-doing the AST to LLVM IR translation.
Replacing an invalidated function by a correct closure, as part of the exit mechanism, mitigates the overhead introduced by the OSR instrumentation in the presence of assumption failures.\\

This report presents an overview of different OSR implementations, and common OSR mechanisms.
In RJIT OSR, we implemented several prototypes to experiment with different techniques present in the literature.
We believe that RJIT could greatly benefit from unguarded OSR points and lazy deoptimizations, once a good interaction with GNUR runtime environment is devised.
We also think that providing transformation primitives that automatically update the values mapping between a base function, and an optimized one, is of great interest.\\

The implementation of a speculative inliner, relying on on-stack-replacement deoptimizations for correctness, enabled to validate our RJIT OSR prototype. 
Function call inlining is not semantically correct in R.
However, thanks to RJIT OSR, we were able to aggressively inline calls throughout the Shootout benchmarks\cite{Shootout}.
One of RJIT's goal is to aggressively and speculatively optimize away costly behaviors inherent to R semantics.
On the other hand, our experience with inlining showed that implementing R-level transformations at the LLVM IR is challenging.
Devising an intermediary representation, closer to R semantics, between GNUR ASTs, and the LLVM IR, that facilitates optimizations, and encoding OSR Exits at that level might be an interesting idea.
LLVM would then only be part of the final compilation steps.\\

We presented interesting future work.
RJIT will soon be able to gather profiling informations about the program at runtime. 
Once these information will become available, RJIT will need to enable runtime re-compilation and optimization of code.
As a result, a better integration of OSR Handler's mechanisms, such as the fresh non-instrumented LLVM IR for each compiled function, might be needed. 
We mentioned attaching non-instrumented IR directly to SEXPs.
Other on-stack replacement mechanisms, such as unguarded points and open OSR points, will also become more attractive.\\